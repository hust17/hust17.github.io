\documentclass[11pt,a4paper]{article}

%\usepackage[tmargin=1in,bmargin=1in,lmargin=1.25in,rmargin=1.25in]{geometry}
\usepackage[margin=1.25in]{geometry}
\usepackage{graphicx}
\usepackage{multicol}
\usepackage{enumitem}
\usepackage[colorlinks=true,linkcolor=blue,citecolor=blue,urlcolor=blue]{hyperref}
\usepackage[sectionbib,numbers]{natbib}
\usepackage[UKenglish]{isodate}
\usepackage{watermark}
\usepackage{transparent}
\usepackage{fix-cm}
\usepackage{fancyhdr}
%\usepackage[utf8]{inputenc}
%\usepackage{pxfonts}
\usepackage[default,osfigures]{opensans}
\usepackage[T1]{fontenc}

\usepackage{sectsty}
%\subsubsectionfont{\itshape}
%\usepackage{draftwatermark}
%\SetWatermarkText{DRAFT}
%\SetWatermarkScale{1}
%\SetWatermarkLightness{0.95}

\setlist{noitemsep}
\parskip 2pt

\usepackage{titlesec}

\titlespacing\section{0pt}{12pt plus 4pt minus 2pt}{4pt plus 2pt minus 2pt}
\titlespacing\subsection{0pt}{12pt plus 4pt minus 2pt}{2pt plus 2pt minus 2pt}
\titlespacing\subsubsection{0pt}{12pt plus 4pt minus 2pt}{2pt plus 2pt minus 2pt}

\begin{document}

\begin{titlepage}
\begin{center}
\vspace{35mm}
{\bf {\fontfamily{phv}\selectfont \fontsize{30}{35}\selectfont HUST-2017:  Fourth International Workshop on HPC User Support Tools.}}\\
\vspace{20mm}
{ {\fontfamily{phv}\selectfont \Huge Call for participation}}\\
\vspace{10mm}
{ {\fontfamily{phv}\selectfont \Huge Held in conjunction with }}\\
{ {\fontfamily{phv}\selectfont \Huge SC17: The International Conference on High
Performance Computing, Networking, Storage and Analysis.}}\\
\vspace{40mm}
{ {\fontfamily{phv}\selectfont \Huge Denver, CO, USA}}\\
{ {\fontfamily{phv}\selectfont \Huge 12 November 2017}}\\
\vspace{25mm}

\includegraphics[width=40mm]{../img/SC17.jpeg}
\hspace{20mm}
\includegraphics[width=40mm]{../img/sighpc-logo.png}
\\
\includegraphics[width=40mm]{../img/lab_logo_blue.png}
\hspace{10mm}
\includegraphics[width=40mm]{../img/PawseyLogoHorizontal.jpg}

\end{center}
\end{titlepage}

\lhead{}
\chead{\bf \fontfamily{phv}\selectfont HUST-2017:  Fourth International Workshop on HPC User Support Tools.}
\rhead{}
\pagestyle{fancy}

\setcounter{page}{2}

\parindent 0pt
\parskip 6pt
\pagebreak
\section{Introduction}

Supercomputing centers exist to drive scientific discovery by supporting researchers in 
computational science fields.  To make users more productive in the complex HPC
environment, HPC centers employ user support teams.  These teams
serve many roles, from setting up accounts, to consulting on math libraries and code
optimization, to managing HPC software stacks.
Often, support teams struggle to adequately support scientists.
HPC environments are extremely complex, and combined with
the complexity of multi-user installations, exotic hardware, and maintaining
research software, supporting HPC users can be extremely demanding.

With the fourth HUST workshop, we will continue to provide a necessary forum for 
system administrators, user support team members, tool developers, policy makers and
end users.  We will provide a forum to discuss support issues and we will
provide a publication venue for current support developments.  Best practices,
user support tools, and any ideas to streamline user support at supercomputing
centers are in scope.

\section{Topics}

Topics of interest include, but are not limited to:

\begin{itemize}
\item defining and customising the user environment
\item software build and installation tools
\item tools and frameworks for using system performance analysis tools
\item workflow and pipeline tools
\item collaboration tools
\item novel environments: cloud, support for Docker.
\item supporting Hadoop and Spark clusters for \emph{Big Data}
\item establishing baseline configuration efforts for HPC
\item software tools for system testing and \emph{monitoring}
\item documentation: creating, maintaining and auto-updating
\end{itemize}

\section{Submission}

We invite authors to submit original, high-quality work with
sufficient background material to be clear to the HPC
community. 
Format: Submissions are limited to 10 pages in the ACM format 
(see http://www.acm.org/sigs/publications/proceedings-templates). 
The 10-page limit includes figures, tables, and your appendices, 
but does not include references, for which there is no page limit. 
Papers should be submitted in PDF format. The margins and font sizes 
should not be modified. We kindly refer authors to the necessary templates [2].

All submissions should be made electronically through the Easychair
website [3].  Submissions must be double blind, i.e., authors should
remove their names, institutions or hints found in references to
earlier work. When discussing past work, they need to refer to
themselves in the third person, as if they were discussing another
researcher's work. Furthermore, authors must identify any conflict of
interest with the PC chair or PC members.

Proceedings will be published in both IEEE Xplore and the ACM Digital
library through collaboration with ACM SIGHPC.

\section{Important dates}
\begin{itemize}
\item Submission deadline: August 18th 2017
\item Workshop paper reviews: by September 19th 2017
\item Acceptance notifications: September 29th 2017
\item Camera-ready papers: October 13th, 2017
\item Workshop: At SC'17, November 12th, 2017
\end{itemize}

\section{Organisers}

\begin{itemize}
\item Christopher Bording, Pawsey Supercomputing Centre, Australia
\item Todd Gamblin, LLNL, USA
\item Olli-Pekka Lehto, Jump Trading, LLC, United Kingdom
\end{itemize}

\subsection{General Chair}
\begin{itemize}
\item Christopher Bording
\end{itemize}
\subsection{Program Chair}

\begin{itemize}
\item Todd Gamblin
\item Olli-Pekka Lehto
\end{itemize}

\subsection{Program Committee}

\begin{itemize}
\item Daniel Ahlin, PDC HPC Center, KTH Royal Institute of Technology, Sweden
\item Erik Engquist, Rice University, USA
\item Wolfgang Frings, Juelich Supercomputing Centre, Germany
\item Todd Gamblin, Lawrence Livermore National Laboratory, USA
\item Andy Georges, Ghent University, Belgium
\item Christopher Harris, Pawsey Supercomputing Centre, Australia
\item Vera Hansper, CSC - IT Center for Science, Finland
\item Paul Kolano, NASA, USA
\item John C. Linford, ParaTools, USA
\item Dave Montoya, Los Alamos National Laboratory, USA
\item Randy Schauer, Raytheon Company, USA
\item Gary B. Skouson, Pacific Northwest Nation Laboratory, USA
\item William Scullin, Argonne National Laboratory, USA
\item Karen Tomko, Ohio Supercomputing Center, USA
\end{itemize}

\end{document}
